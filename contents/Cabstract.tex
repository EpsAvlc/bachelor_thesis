% !Mode:: "TeX:UTF-8"
\begin{Cabstract}{无人车}{障碍物检测}{激光雷达}{点云}{多传感器融合}
近年来,无人车自动驾驶与导航技术的蓬勃发展催生了无人车产业的蓬勃兴起。激光雷达传感器(LiDAR)能够为车辆提供详尽的周围环境感知,因此被广泛应用于无人驾驶的场景中。激光雷达产生的数据为点云,而影响点云分辨率的参数为激光雷达的线数。低线数激光雷达获得的点云的垂直分辨率较低,影响无人车环境感知的效果;而高线数激光雷达成本高昂,直接制约了无人驾驶技术的普及。因此,如何有效的利用低线数雷达提供的数据进行环境感知,成为了目前在无人车领域较为热门的话题。

为了解决上述问题,本文提出了一种新颖的三维感知传感器机构,该机构通过无刷电机驱动曲柄连杆,为低线数的激光雷达提供垂直方向上的往复旋转运动;在此基础上,融合注册激光雷达在每次往复运动中发布的点云,来提升点云垂直方向上的分辨率;针对由于激光雷达的运动而产生的点云的运动畸变,本文提出一种帧内多次线性插值的方法,通过对激光雷达的每帧点云进行时间上的分段线性插值,有效地矫正了因三维感知机构运动而导致的点云的运动畸变。

此外,为了能够充分利用无人车各个传感器的数据进行三维障碍物的检测,本文还通过标定激光雷达与相机的外参,融合了激光雷达与相机的数据,对三维障碍物进行了检测与定位;同时利用激光雷达的点云信息对相机的目标检测结果做了优化,提高了基于图像的目标检测方法的查准率。

为了验证本文提出的三维感知机构的可靠性,本文首先利用Gazebo对三维感知机构进行了仿真,随后在无人车实验平台上利用上述的三维感知机构进行了三维建图与点云的障碍物分割的实验,并且检验了相机与激光雷达的标定效果。实验结果表明,本文提出的三维感知机构能够有效地提高激光雷达的垂直分辨率。通过三维感知机构融合注册后的点云,其垂直分辨率可以达到64线激光雷达的三到四倍,对于三维建图以及无人车环境下的三维障碍物的检测都有着重要的意义。
\end{Cabstract}
