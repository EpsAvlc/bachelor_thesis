% !Mode:: "TeX:UTF-8"


\chapter{点云的多帧融合与激光雷达和相机标定}
根据本文第二章所提到的机构,能够将三维激光雷达在其yaw角方向上提供一个有规律的正弦往复运动。本文提出该机构的主要目的为将激光雷达在时间轴上的多帧点云进行融合,进而增加激光雷达在竖直方向的分辨率,达到近似于给激光雷达增加线数的效果。本章将对上述机构得到多帧激光雷达点云进行融合,并且在目前机构的基础上进行激光雷达与相机的标定,从而使得融合后的点云能够应用于第四章提到的基于视觉与激光融合的三维障碍物检测方法。

\section{激光雷达点云的多帧融合}

\subsection{一种朴素的多帧融合策略}

最为直观的策略就是,读取绝对值编码器返回的角度$\alpha$,将激光雷达每帧点云沿着航向角方向旋转$-\alpha$的角度,然后注册多帧的激光雷达点云并发布。

在实际的实现过程中,编码器返回角度的频率约为30Hz,而点云发布的频率约为10Hz,在将点云旋转$-\alpha$角度时,对$\alpha$角进行了线性插值以便获得更加精确的结果。同时,根据计算曲柄机构的角度是增加还是减少,来判断曲柄机构的运动方向,并且将曲柄机构运动角度为一个正弦周期内的点云融合为一帧新的点云输出。

然而这种朴素的融合策略在实际中效果不好,体现在融合后的点云所显示的物体轮廓失真严重,原因是没有考虑激光雷达的运动对激光雷达点云生成的影响。

\subsection{点云的运动畸变的形成与矫正}
在激光雷达点云的多帧融合中,如果只是进行简单的历史点云叠加(如上文所示),那么融合后的点云相较于真实情况会有很严重的失真,其原因就在于第二章所提到的三维感知机构在给激光雷达在偏航角方向上的往复运动时,点云会产生不可忽视的运动畸变。本章节首先介绍什么是激光雷达点云的运动畸变,然后提出一种通过插值的方式矫正激光雷达的运动畸变。

\subsubsection{点云的运动畸变}
激光雷达的点云的形成本质上是由激光雷达内部的多个激光测距器将一个旋转周期内的各个测量值记录下来并同时发布后得到的。因此点云中的每个点并不是在同一时刻被测量出来的。如果激光雷达在测量的过程中也在运动,那么激光雷达的点云可能会发生畸变\citeup{hong2010vicp}。

下面以二维激光雷达为例,介绍激光雷达点云运动畸变的形成。

\begin{pics}[htbp]{激光雷达运动畸变}{distort}
    \addsubpic{ground truth}{width=0.3\textwidth}{distortion_groundtruth}
    \addsubpic{采集得到的数据}{width=0.3\textwidth}{distortion}
\end{pics}

图\ref{distortion_groundtruth}中的黑色的线条表示二维激光雷达处在的真实环境的轮廓图,箭头表示二维激光雷达的运动方向。图\ref{distortion}中的蓝色的线条表示二维激光雷达的原始数据。注意到其已经发生了畸变,因为二维激光雷达内的激光测距器通过逆时针的方向旋转,使得右上角的数据优先得到,而左上角的数据在激光雷达向箭头方向运动了一段距离之后才进行测量,自然导致了运动畸变的产生。

值得一提的是上图表示的二维激光雷达发生的运动畸变是当激光雷达运动方向为水平运动方向时造成的,而当激光雷达在空间中有垂直方向的旋转运动时,其造成的运动畸变远比水平运动严重。这是因为激光雷达旋转一周的时间普遍在0.1秒左右,其水平运动的距离往往很小可以忽略不计。而当激光雷达有竖直方向的旋转时,即使在0.1s内只有2度的航向角的旋转(这在本文的机构中并不算很快),在测量20m处的物体时,其运动造成的点云畸变可使得点云的同一线上的第一个点与最后一个点的垂直相差将近70cm。

综上所述,对于第二章所述的机构,由于其施加了在激光雷达航向角方向上的旋转,因此导致其在竖直方向上的运动畸变不可忽视,从而简单的叠加点云会导致在做激光雷达的物体检测时的失真。

\subsubsection{运动畸变的矫正}

\begin{pics}[htbp]{畸变消除前后对比}{distort_remove}
    \addsubpic{轿车(消除畸变前)}{width=0.3\textwidth}{car_before}
    \addsubpic{轿车(消除畸变后)}{width=0.329\textwidth}{car_after}
\end{pics}

\subsection{矫正运动畸变后的多帧融合策略}

\section{激光雷达与相机的标定}

\section{本章小结}

