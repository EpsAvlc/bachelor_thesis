% !Mode:: "TeX:UTF-8"

\chapter{绪论}
\section{研究工作的背景与意义}
近几年来,自动驾驶技术取得了长足的进步,而其中关键的技术就是多传感器的环境感知与融合。环境感知的一个重要环节便是障碍物检测。目前,虽然基于图像的障碍物检测已经取得了卓有成效的进步,然而相较于三维障碍物检测,二维障碍物检测有以下缺陷:
\begin{enumerate}
    \item 基于单目相机的障碍物检测没有尺度信息,无法恢复出目标的三维坐标。
    \item 基于双目相机的障碍物检测,当基线较短时,测量距离较长(5m以上)的物体时计算出来的距离信息很不准确,而当基线较长时,近处物体的检测又容易出现在两个相机的视野盲区之中,从而导致无法三角化而得出距离信息。
\end{enumerate}

基于上述原因,越来越多的目光聚焦在了基于激光雷达(LiDAR)的三维障碍物检测。LiDAR是Light Detection And Ranging的缩写,中文译作“激光探测与测量”,一般指多线数的三维激光雷达传感器。相较于相机图像,激光雷达的点云拥有以下几点优势:
\begin{enumerate}
    \item 测量范围广。目前的激光雷达的测量有效距离基本都在0.5-100米左右,远高于双目相机三角测距的适用范围。
    \item 测量精度高。激光雷达的测距误差可达厘米级,同样优于双目相机的测距结果。
\end{enumerate}

目前最常见的旋转式激光雷达,其本质是多个激光束旋转后对每个时刻的测距结果进行保留与叠加,最后再以点云的形式发布出去。决定激光雷达的分辨率的一个参数为其激光束的个数,一般称之为激光雷达的线数。目前常用的激光雷达线数有16线、32线、64线等,其中由于低线数的激光雷达生成的点云在测量远距离物体时密度较低。举例来说,当使用16线激光雷达检测到20米处的障碍物时,其16线激光束两两之间的距离可以达到70cm,相当于检测20m处的人时,只能够有两线激光束能够返回距离。因此,低线数激光雷达点云的稀疏性较大地制约了三维障碍物检测任务的准确率。

通常在自动驾驶的无人车系统中会在车的四周装上多个16线的激光雷达进行点云融合,或者直接采用线数更高的激光雷达来做障碍物检测的任务。然而目前三维激光雷达造价不菲,无人车系统中光是64线激光雷达的成本就奖金十万美金,如此高昂的成本在一定程度上限制了无人驾驶汽车的普及与推广。

鉴于上述存在在问题,本文希望能够提出一种基于多帧融合的低线数激光雷达感知机构,使其能够通过增加一个在垂直方向的往复运动,并将该机构上激光雷达的多帧点云融合发布来提高三维点云的稠密性,借而解决低线数激光雷达在障碍物检测问题上由于点云的稀疏性而造成的困难。并且,本文还希望通过融合上述机构发布的点云信息以及相机的图像信息,发挥各个传感器的优势从而提高三维障碍物检测任务的准确率与实现三维障碍物的多类别检测。

\section{无人车三维障碍物检测的国内外研究历史与现状}
根据本文的主要研究方向,下面将对基于激光雷达点云的三维目标分割与检测(3D object segmentation and detection)的研究现状进行调研。

针对三维障碍物的分割问题,目前主要有三类方法来实现

\section{本文的主要贡献与创新}

\section{本论文的结构安排}
