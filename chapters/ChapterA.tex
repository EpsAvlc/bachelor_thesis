% !Mode:: "TeX:UTF-8"

\chapter{绪论}
\section{研究工作的背景与意义}
近几年来,自动驾驶技术取得了长足的进步,而其中关键的技术就是多传感器的环境感知与融合。在这些传感器中,激光雷达(Lidar)凭借其准确、稳定的特点,在自动驾驶中具有举足轻重的地位,被称之为无人驾驶汽车的眼睛。然而,低线数激光雷达

(例如:FEKO、Ansys 等)的
核心算法。由文献\cite{feng997he,clerc2010discrete,xiao2012yi}可知……

……
\section{无人车三维障碍物检测的国内外研究历史与现状}
时域积分方程方法的研究始于上世纪60 年代,C.L.Bennet 等学者针对导体目
标的瞬态电磁散射问题提出了求解时域积分方程的时间步进(marching-on in-time,
MOT)算法\citeup{zhong1994zhong}。……

……
\section{本文的主要贡献与创新}
本论文以时域积分方程时间步进算法的数值实现技术、后时稳定性问题以及
两层平面波加速算法为重点研究内容,主要创新点与贡献如下:

……
\section{本论文的结构安排}
本文的章节结构安排如下:

……
