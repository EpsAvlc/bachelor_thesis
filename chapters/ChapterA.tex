% !Mode:: "TeX:UTF-8"

\chapter{绪论}
\section{研究工作的背景与意义}
近几年来,自动驾驶技术取得了长足的进步,而其中关键的技术就是多传感器的环境感知与融合。环境感知的一个重要环节便是障碍物检测。目前,虽然基于图像的障碍物检测已经取得了卓有成效的进步,然而相较于三维障碍物检测,二维障碍物检测有以下缺陷:
\begin{enumerate}
    \item 基于单目相机的障碍物检测没有尺度信息,无法恢复出目标的三维坐标。
    \item 基于双目相机的障碍物检测,当基线较短时,测量距离较长(5m以上)的物体时计算出来的距离信息很不准确,而当基线较长时,近处物体的检测又容易出现在两个相机的视野盲区之中,从而导致无法三角化而得出距离信息。
\end{enumerate}
基于上述原因,越来越多的目光聚焦在了基于激光雷达(LiDAR)的三维障碍物检测。LiDAR是Light Detection And Ranging的缩写,中文译作“激光探测与测量”,一般指多线数的三维激光雷达传感器。相较于相机图像,激光雷达的点云拥有以下几点优势:
\begin{enumerate}
    \item 测量范围广。目前的激光雷达的测量有效距离基本都在0.5-100米左右,远高于双目相机三角测距的适用范围。
    \item 测量精度高。激光雷达的测距误差可达厘米级,同样优于双目相机的测距结果。
\end{enumerate}
目前最常见的旋转式激光雷达,其本质是多个激光束旋转后对每个时刻的测距结果进行保留与注册,最后再以点云的形式发布出去。决定激光雷达的竖直方向分辨率的参数为其激光束的数量,一般称之为激光雷达的线数。目前常用的激光雷达线数有16线、32线、64线等,其中由于低线数的激光雷达生成的点云在测量远距离物体时竖直方向分辨率较低。因此,低线数激光雷达点云的稀疏性较大地制约了三维障碍物检测任务的准确率。

通常在自动驾驶的无人车系统中会在车的四周装上多个16线的激光雷达进行点云融合,或者直接采用线数更高的激光雷达来做障碍物检测的任务。然而目前三维激光雷达造价不菲,无人车系统中光是64线激光雷达的成本就奖金十万美金,如此高昂的成本在一定程度上限制了无人驾驶汽车的普及与推广。

鉴于上述存在在问题,本文希望能够提出一种基于多帧融合的低线数激光雷达感知机构,使其能够通过增加一个在垂直方向的往复运动,并将该机构上激光雷达的多帧点云融合发布来提高三维点云的稠密性,借而解决低线数激光雷达在障碍物检测问题上由于点云的稀疏性而造成的困难。并且,本文还希望通过融合上述机构发布的点云信息以及相机的图像信息,发挥各个传感器的优势从而提高三维障碍物检测任务的准确率与实现三维障碍物的多类别检测。

\section{国内外研究历史与现状}
根据本文的主要研究方向,下面将对面向无人车的三维障碍物检测方面的研究展开调研。
目前,面向无人车的三维障碍物检测主要可以分为基于相机的检测方法、基于雷达的检测方法和多传感器融合的检测方法这三种类型。
\subsection{基于相机的检测方法}
基于相机的检测方法目前主要是以基于深度学习的物体检测为主。随着 ImageNet数据集\citeup{Li2010lsvrc}的建立以及大规模目标检测竞赛的进行,深层卷积网络被成功运用在了图像识别与物体检测领域\citeup{7112511},而通过深度学习来实现物体的识别与分类,正是目前无人车障碍物检测的主流方法。近些年来还有一些基于单目图像的端到端的深度估计的网络也被提出\citeup{10.1007/978-3-319-46484-8_45},能够直接单目图像对图像的每个像素进行深度估计。不过目前这种方法得到的深度估计还不够精确,并且相当依赖于数据集,泛化能力不够强,因此直接利用这种深度估计进行障碍物检测的方法还比较少。
\subsection{基于激光雷达的检测方法}
激光雷达通过发射激光并测量其返回时间,得到距离信息,测量距离远、速度快、误差小并且分辨率高。基于激光雷达的障碍物检测一般可分为基于模型与无模型两种类别。基于模型的检测方法\citeup{rabbani2005efficient}通过先验知识构建模型来同时进行点云的检测与分类,但是计算量巨大并且很难做到实时。无模型的方法则首先采用模型估计地面\citeup{montemerlo2008junior}来去除地面点云,为了减 少计算量,将剩余点云投影到地面\citeup{behley2013laser}或者一个虚拟的平面上生成图像(称之为深度图(Range image))\citeup{korchev2013real},之后再从图像上进行障碍物的检测与分割。

另外亦有国外研究提出3D检测网络将特征提取和边界框预测统一到单个阶段的端到端可训练深度网络中,通过深度学习同时解决点云的检测与分类问题\citeup{zhou2018voxelnet}。其深度学习网络输入为点云,通过对点云预处理和卷积后得出障碍物的包围盒以及类别。

\subsection{多传感器融合的检测方法}
在实际应用中,无人驾驶汽车障碍物检测的主流方法主要是利用雷达与图像融合\citeup{zhang2014vehicle}的方法。单纯利用相机缺点是获取准确的三维信息比较困难,并且受环境光线的影响比较大;而对于激光雷达,由于点云的稀疏不一致性,使得单纯依靠点云进行目标检测的准确率较低。而将激光雷达与相机结合,则可以取长补短,实现对周边物体的检测分类与定位。

\section{本文的主要贡献与创新}
针对低线数激光雷达在三维障碍物检测问题中因为点云的稀疏性与不一致性给三维目标检测带来的困难,本文提出了一种新颖的三维感知机构,通过给激光雷达提供偏航角方向上的往复运动,并且融合多帧点云来增加激光雷达点云在竖直方向上的分辨率,同时通过帧内多次线性插值来矫正因机构运动而带来的点云的运动畸变。

同时,本文还通过对相机与激光雷达的标定,融合了相机图像与激光雷达的传感器数据信息,借由相机的目标检测算法来对物体进行检测,通过激光雷达的点云信息来对检测出的三维障碍物进行定位,并且利用点云信息优化了视觉的目标检测的结果。

\section{本论文的结构安排}
本文的结构安排如下:
\begin{enumerate}
    \item 第一章为绪论,介绍面向无人车的三维障碍物检测的研究背景与本文的主要研究内容。
    \item 第二章为三维感知机构的设计,介绍了本文提出的一种三维感知机构的机械、电路、软件方面的设计。
    \item 第三章介绍了点云的畸变矫正以及相机与激光雷达的标定,为后文的相机与激光雷达的数据融合提供外参。
    \item 第四章介绍了相机与激光雷达的融合,利用相机图像对障碍物进行检测与分类,同时利用激光雷达的点云信息来对障碍物进行定位。
    \item 第五章根据前文提到的三维感知机构,设计了仿真实验与基于点云投影到深度图像的物体分割。
\end{enumerate}