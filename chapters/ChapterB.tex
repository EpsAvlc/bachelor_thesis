% !Mode:: "TeX:UTF-8"

\chapter{三维感知传感器机构设计}
为了能够解决低线数激光雷达在障碍物检测问题上由于点云的稀疏性而造成的困难,本章提出了一种三维感知传感器机构,通过增加三维激光雷达在垂直方向的往复旋转,同时融合多帧激光点云,来增加激光雷达在铅垂方向上的分辨率,实现类似于高线数激光雷达的稠密点云。

\section{三维感知传感器的机械结构设计}
\begin{pics}[htbp]{机构总图}{structure_all}
    \addsubpic{solidworks渲染图}{width=0.5\textwidth}{structure_sw}
    \addsubpic{实物图}{width=0.5\textwidth}{structure}
\end{pics}
\subsection{运动原理}
本章所述的机构结构如图\ref{structure}所示,其中3508电机提供驱动转矩,曲柄连杆装置将电机的旋转运动转化为激光雷达底座在航向角(yaw)方向上的往复运动,同时绝对值磁编码器记录激光雷达在航向角上的角度变化,以供多传感器融合时使用。

\section{三维感知传感器的电路设计}

\begin{pics}[htbp]{驱动器与传感器}{sensors}
    \addsubpic{C620 电调}{width=0.2\textwidth}{modulation}
    \addsubpic{M3508无刷电机}{width=0.15\textwidth}{3508}
    \addsubpic{磁编码器}{width=0.2\textwidth}{encoder}
    \addsubpic{RM A型开发板}{width=0.3\textwidth}{dev_board}
\end{pics}

\subsection{电路拓扑}

\subsection{驱动器}
该三维感知机构采用的驱动器为DJI C620电调,如图\ref{modulation}所示。该电调采用CAN指令或者PWM两种调节方式,最大电流可达20A。

\subsection{执行机构}
该三维感知机构采用的执行机构为DJI M3508无刷电机,如图\ref{3508}所示。该电机持续最大转矩可达3Nm,满足曲柄机构的结构受力要求。

\subsection{传感器}

\subsubsection{角度传感器}
该三维感知机构采用的角度传感器为傲蓝13线磁编码器,如图\ref{encoder}所示。该编码器采用RS485方式通信,其单圈分辨率为8192cpr, 精度为$\pm 0.1$度。

该编码器为绝对值式编码器,其相对于增量式编码器不同点在于,增量式编码器以上电时的位置为零点,每次使用都要机械对位;而绝对值式编码器能够记录机构的唯一位置,即单圈内编码器的每一个示值,都唯一对应了空间中机构的位置与角度。考虑到我们曲柄连杆机构的特性,显然绝对值式编码器更加符合我们的要求。

\subsubsection{激光雷达}
\pic[htbp]{速腾16线激光雷达}{width=0.4\textwidth}{robosense}
该三维感知机构采用的激光雷达为速腾聚创的RS-LiDAR-16,如图\ref{robosense}所示。该激光雷达为16线激光雷达,其测距范围为50cm-150m,精度误差为$\pm 2cm$。垂直视场角为30度,其角分辨率为2度;水平视场角为360度,其角分辨率为0.09-0.36度(对应的点云频率为5Hz-20Hz)。

\subsection{主控板}
该三维感知机构采用的主控板为DJI Robomaster A型开发板,如图\ref{dev_board}所示。该开发板具备类型丰富的接口,包括12V、5V、3.3V电源接口、CAN接口、UART接口、可变电压PWM接口、SWD接口等。同时该开发板拥有电源输入的防反接、过压保护、缓启动、12V电源输出过流保护、PWM端口的ESD等多重保护。

\section{三维感知传感器的软件设计与运动控制}

\section{本章小结}