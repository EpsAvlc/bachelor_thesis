% !Mode:: "TeX:UTF-8"

\chapter{全文总结与后续工作展望}

\section{全文总结}
在无人车自动驾驶领域,如何利用好激光雷达来进行环境感知与障碍物检测一直是一个热门的话题。低线数激光雷达在环境感知时返回的信息较少,很难直接根据点云信息对物体进行识别,而高线数激光雷达又造价高昂,极大限制了无人驾驶的成本。因此,本文立足于解决低线数激光雷达由于线数较少而较难根据点云信息检测障碍物的问题,设计了一种新颖的三维感知机构,能够通过给低线数激光雷达提供航向角上的往复运动来增加激光雷达竖直方向上的分辨率;同时,本文亦融合了相机与激光雷达点云的数据信息,利用图像对障碍物进行了分类与分割,结合点云完成了对障碍物的定位。本文完成的主要研究工作有:
\begin{enumerate}
    \item 提出一种新颖的三维感知机构,该机构通过无刷电机驱动曲柄摇杆机构来给底座上的激光雷达提供航向角上的往复运动,并通过融合多帧激光雷达的点云来增加低线数激光雷达在竖直方向上的点云的分辨率。在融合的过程中,本文通过帧内多次线性插值来矫正因激光雷达往复运动而造成的运动畸变,使得融合后的点云能够更加真实地反应环境的情况。
    \item 融合了相机图像与激光雷达点云的信息,对三维障碍物进行了分割、分类以及定位。首先利用目标检测网络YOLO来对图像进行推断,对障碍物进行了分割与分类;随后,根据投影到相机上的点云信息,计算出了待检测目标的三维位置信息;最后,利用点云的三维位置信息来解决YOLO在低置信度阈值下的误检测问题,提高了YOLO在低执行度阈值下的目标检测的查准率。
\end{enumerate}

综上,本文面向无人车领域,提出了一种新颖的利用低线数激光雷达多帧融合来进行环境感知的方式。通过融合多帧点云,本文提出的机构能够提供类似于高线数激光雷达的丰富的点云信息,同时利用该点云信息,结合相机图像信息进行了三维障碍物的识别与定位。

\section{后续工作展望}
有关无人车自动驾驶环境感知的研究近几年发展迅速,在本文研究工作的基础上,仍有
以下方向值得进一步研究:
\begin{enumerate}
    \item 在本文提出的多帧融合三维感知机构中,如果障碍物运动速度较快,则由于多帧融合,在融合后的点云中会有由于障碍物运动而产生的拖影。在三维建图时,如何消除运动物体产生的点云来建立只有静态物体的地图是一个值得深入研究的问题。一个可行的办法是引入八叉树地图(OctoMap),对于传感器观测到的障碍物,归入八叉树地图,利用贝叶斯滤波来滤除动态物体;随后再用八叉树地图来恢复出没有动态障碍物的点云。
    \item 在相机与传感器的融合中,目前使用的YOLO卷积网络的输入为图像,其为$m\times n\times 3$的有序矩阵信息。而通过将融合后的点云投影,则能够为图像的像素提供第四维信息,即深度信息,如果将四维像素信息输入卷积网络,则有希望大大提高YOLO目标检测的查准率与查全率。而且由于本文所得的激光雷达进行了竖直方向的分辨率的提升,使得投影到图像上的点云密度能够较好的和像素相匹配。
\end{enumerate}
总而言之,本文提出的三维感知机构实现了低线数激光雷达点云在竖直方向上的稠密化,其对于三维稠密地图的建立以及三维目标检测的任务都有较为重要的意义,基于此平台可以未来可以开展许多有关三维环境感知感知、三维建图等方面的工作。